%% Template copyright 2007 Xavier Danaux (xdanaux@gmail.com).
%
% This work may be distributed and/or modified under the
% conditions of the LaTeX Project Public License version 1.3c,
% available at http://www.latex-project.org/lppl/.

% To compile: lualatex (NOT luatex, luahbtex, latex)
% So far, I've had most success by installing miktex and then
% having it install lualatex.


\documentclass[roman]{moderncv}
\usepackage{fontspec}
% \usepackage{fontawesome}
\usepackage{enumitem}

% \setmainfont{URW Gothic}

% moderncv themes
\moderncvtheme[green]{casual}

% optional arguments are 'blue'
% (default), 'orange', 'red', 'green',
% 'grey' and 'roman' (for roman fonts)
% {casual, classic, banking, oldstyle}

% character encoding
\usepackage[utf8]{inputenc}

% adjust the page margins
\usepackage[scale=0.8]{geometry}
\recomputelengths

% Emphasize terms
\renewcommand{\emph}[1] {\textbf{\textit{#1}}}

% personal data
\firstname{Liudvikas}
\familyname{Akelis}
\title{Curriculum Vitae}
\address{Vilnius, Lithuania}
\mobile{+37069053684}
\email{liudvikas.akelis@gmail.com}
\extrainfo{\url{https://github.com/liudvikasakelis}}

% uncomment to suppress automatic page numbering for CVs longer than one page
%\nopagenumbers{}


\begin{document}
\maketitle

\section{About me}
\cvitem{}{A chemist by training, I was fascinated by computers from an early age. After becoming interested in statistics, I pivoted my career to data science in 2017. Since then, I have worked in many data areas - from determining data requirements to data engineering to creating decisions from that data. I enjoy working in teams, pushing into git, cycling and bouldering.}

\section{Work experience}


\cventry{Apr 2025 - current}{Data Engineer}{Saige Consulting Europe}{}{}{
  Building data warehouses for external clients using Databrics, Azure Datafactory and Python.
  \begin{itemize}
  \item Migrated an ongoing project to GIT without development downtime
  \end{itemize}
}

\cventry{Apr - Oct 2024}{Data Engineer}{NASDAQ, Vilnius}{}{}{
  Maintained a small internal data warehouse for the Internal Audit team.
  \begin{itemize}
  \item Researched (benchmarked) and initiated a migration to Python. This involved both re-implementing some functions in Python and interoperating with functions whose migration was postponed.
  \item Introduced unit testing. Especially during a migration, it's very useful to have assurance that data transformations are migrated faithfully.
  \item Onboarded new data sources, fixed old ones.
  \end{itemize}
}

\cventry{May 2020 - Nov 2023}{Decision Scientist}{Vinted, Vilnius}{}{}{
  Helped make decisions more data-based in full-stack development teams
  \begin{itemize}
  \item Proposed, defined  and implemented various metrics for the team. OKRs at the top level, but also their decomposition into drivers.
  \item Contributed to exposing our data to the rest of the company by developing data models, developing nightly jobs to calculcate them, and finally expose this data through \emph{Looker}.
  \item Organized and overlooked AB experiments on both internal and customer-facing features: proposed and defined measures and how to measure them, calculated sample sizes before the experiments and result statistical significance after.
  \end{itemize}
}

\cventry{Nov 2017 - May 2020}{Data analyst}{UAB Scorify, Vilnius}{}{}{
  A position that expanded as I spent more time in the company and got to know
  the infrastructure better.
  \begin{itemize}
  \item Started with tasks in data extraction, such as imports from raw
    sources and web scraping.
  \item Moved on to building predictive models for projects, working mostly with \emph{Tensorflow} and \emph{xgboost} - respectively convolutional neural networks and random forests.
  \item On the side, worked to revive and took on administrating one of our
    older Linux servers.
  \item Developed a few internal webapps, using the amazing \emph{Shiny}
    framework, which we used to spread our data insights within the company.
  \end{itemize}
}

\newpage


\section{Technical skills}
\cvitem{}{
  \begin{itemize}
  \item For data analysis and modelling, I absolutely love \emph{R} with 5+ years of experience. Similar amount of experience with \emph{Python}.
  \item 5+ years of SQL (MySQL, Impala, BigQuery, MSSQL, SQLite). For complex queries, I however prefer frameworks (\emph{ibis}, \emph{dbplyr})
  \item Working experience with Databricks on Azure, Azure Datafactory
  \item 3 years of \emph{(Scala / Python) Apache Spark}
  \item \emph{git}
  \item Technologies I don't consider my main strengths:
    \begin{itemize}
    \item some \emph{dbt} experience (helped deploy a small project)
    \item \emph{guile scheme} and \emph{Emacs Lisp}
    \item \emph{Linux} and \emph{bash}
    \item \emph{C}
    \item Basic knowledge of how Web works (\emph{HTML}, vanilla \emph{JS}, \emph{CSS}).
    \item \emph{Docker} containers and other \emph{virtualization} stack.
    \end{itemize}
  \end{itemize}
}


\section{General skills}
\cvitem{}{
  \begin{itemize}
  \item Native Lithuanian
  \item Fluent English
  \item Very basic French, German
  \item Drivers License
  \end{itemize}
}

\section{Education}
\cventry{2011-2015}{Vilnius University}{B.Sc. Chemistry}{}{Vilnius, Lithuania}
        {}
\cventry{2016}{Coursera}{3 month course in data science}{}{}
        {3 courses, 1 month each, covering basics of R, basic data analysis
          and visualisations.}
\closesection{}                   % needed to renewcommands

\end{document}
